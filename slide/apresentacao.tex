\documentclass[aspectratio=169]{beamer}

\usepackage[utf8]{inputenc}
\usepackage[T1]{fontenc}
\usepackage[brazil]{babel}
\usepackage{amsfonts,amsmath,oldgerm}
\usepackage{listings}
\usepackage{xcolor}
\usepackage{tikz}
\usepackage{colortbl}

% --- THEME AND COLORS ---
\usetheme{_statale}
\usefonttheme[onlymath]{serif}

% Cores do VS Code
\definecolor{codegreen}{rgb}{0,0.6,0}
\definecolor{codegray}{rgb}{0.5,0.5,0.5}
\definecolor{codepurple}{rgb}{0.58,0,0.82}
\definecolor{backcolour}{rgb}{0.95,0.95,0.92}

\lstset{
    language=C,
    backgroundcolor=\color{backcolour},   
    commentstyle=\color{codegreen},
    keywordstyle=\color{magenta},
    numberstyle=\tiny\color{codegray},
    stringstyle=\color{codepurple},
    basicstyle=\ttfamily\scriptsize,
    breakatwhitespace=false,         
    breaklines=true,                 
    captionpos=b,                    
    keepspaces=true,                 
    numbers=none,                    
    showspaces=false,                
    showstringspaces=false,
    showtabs=false,                  
    tabsize=2,
    frame=single
}

% --- METADATA ---
\title{Sistema de Gerenciamento de Cozinha Transacional}
\subtitle{Estrutura de Dados - UFS}
\course{Ciência da Computação}
% Autor curto para o rodapé não estourar
\author[Grupo 6: Helen, João, Breno, Gabriel, Caio, Lucas]{Helen Bispo, João Simões, Breno Santos, Gabriel Bernardo, Caio Leite, Lucas Cavalcante}
\IDnumber{2026}

% --- LOGOS ---
\pgfdeclareimage[height=0.9cm]{dcomplogo}{assets/dcomp.png}
\renewcommand{\@logo}{dcomplogo}

% Fundo do slide
\titlebackground*{assets/background}

% --- FOOTER OVERRIDE ---
\setbeamertemplate{footline}{%
  \begin{beamercolorbox}[wd=\textwidth,ht=6mm,dp=3.5mm,rightskip=1cm,leftskip=1cm]{footline}
    \usebeamerfont{footline}
    \insertframenumber/\inserttotalframenumber \hspace{0.5cm}
    \textbf{\insertshortauthor} \hfill \inserttitle
  \end{beamercolorbox}
}

\begin{document}

% Inserimos a logo da UFS no campo de data para aparecer somente na capa
\date{18 de fevereiro de 2026
    \begin{tikzpicture}[remember picture,overlay]
        \node[anchor=south east, xshift=-0.5cm, yshift=1.2cm] at (current page.south east) {
            \includegraphics[height=1.2cm]{assets/ufs.png}
        };
    \end{tikzpicture}
}

\maketitle

% Remove a logo do DCOMP (headline) dos proximos slides
\setbeamertemplate{headline}{}

\footlinecolor{maincolor}

% --- SLIDE 1: O Que Faz? ---
\begin{frame}{Definição do Sistema}
    \begin{columns}
        \begin{column}{0.6\textwidth}
            \begin{itemize}
                \item \textbf{Objetivo:} Gerenciar o fluxo operacional de uma cozinha (Insumos $\rightarrow$ Receitas $\rightarrow$ Estoque $\rightarrow$ Pedidos).
                \item \textbf{Diferencial:} Implementação de um motor transacional que garante consistência.
                \item \textbf{Híbrido:} Núcleo em C para eficiência e Interface Web para usabilidade.
            \end{itemize}
        \end{column}
        \begin{column}{0.4\textwidth}
            \begin{alertblock}{Conceito Chave}
                Operação \textit{All-or-Nothing}: Ou o pedido é processado integralmente, ou o estoque é restaurado.
            \end{alertblock}
        \end{column}
    \end{columns}
\end{frame}

% --- SLIDE 2: Mapeamento de Estruturas (TABELA) ---
\begin{frame}{Mapeamento de Estruturas de Dados}
    \small
    \begin{table}
        \centering
        \begin{tabular}{|l|l|l|}
            \hline
            \rowcolor{maincolor} \textcolor{white}{\textbf{Estrutura}} & \textcolor{white}{\textbf{Aplicação no Projeto}} & \textcolor{white}{\textbf{Justificativa Técnica}} \\ \hline
            \textbf{Structs} & Entidades (Insumo, Receita) & Organização de dados heterogêneos \\ \hline
            \textbf{Arrays} & Catálogo e Estoque & Acesso aleatório $O(1)$ por índice \\ \hline
            \textbf{Ponteiros} & Alocação Dinâmica e Listas & Gestão eficiente de memória (Heap) \\ \hline
            \textbf{Lista} & Itens de uma Receita & Tamanho dinâmico e imprevisível \\ \hline
            \textbf{Fila} & Sequência de Pedidos & Ordem de chegada (\textbf{FIFO}) \\ \hline
            \textbf{Pilha} & Rollback Transacional & Desfazer operações (\textbf{LIFO}) \\ \hline
        \end{tabular}
    \end{table}
\end{frame}

% --- SLIDE 3: Structs e Ponteiros ---
\begin{frame}[fragile]{Modelagem com Structs e Ponteiros}
    A estrutura \texttt{NoIngrediente} é a base das receitas, unindo o ID do catálogo à quantidade necessária via encadeamento de ponteiros.
    \begin{lstlisting}[language=C, title=Fragmento: core/ingredientes.h]
typedef struct NoIngrediente {
    int id_ingrediente;         // FK para o catalogo global
    float quantidade;           // Valor real (kg, l, un)
    struct NoIngrediente* prox; // O coracao da Lista Ligada
} NoIngrediente;
    \end{lstlisting}
    \textbf{Destaque:} O uso de ponteiros permite que receitas cresçam sem necessidade de blocos de memória contíguos.
\end{frame}

% --- SLIDE 4: Pilha e Rollback ---
\begin{frame}[fragile]{O Mecanismo de Rollback (Pilha LIFO)}
    Durante o processamento, utilizamos uma pilha para rastrear cada retirada parcial no estoque.
    \begin{columns}
        \begin{column}{0.45\textwidth}
            \begin{lstlisting}[language=C, title=Logica de Rollback (rollback.c)]
void rb_desfazer(PilhaRollback* p, Estoque* e) {
    int id; float qtd;
    while (rb_pop(p, &id, &qtd)) {
        // Restauracao atomica
        est_adicionar(e, id, qtd); 
    }
}
            \end{lstlisting}
        \end{column}
        \begin{column}{0.5\textwidth}
            \begin{itemize}
                \item \textbf{LIFO:} O último ingrediente retirado é o primeiro a ser devolvido.
                \item \textbf{Confiabilidade:} Garante que o estoque permaneça íntegro mesmo em falhas críticas.
            \end{itemize}
        \end{column}
    \end{columns}
\end{frame}

% --- SLIDE 5: Persistencia em Arquivo ---
\begin{frame}[fragile]{Persistência em Arquivos de Texto}
    O sistema mantém o estado entre execuções usando arquivos formatados (CSV-like), simulando um banco de dados.
    \begin{lstlisting}[language=C, title=Serializacao: core/persistencia.c]
int pers_salvar_catalogo(const CatalogoIngredientes* cat) {
    FILE* f = fopen("data/ingredientes.txt", "w");
    for (size_t i = 0; i < cat->qtd_atual; i++) {
        fprintf(f, "%d;%s;%s\n", cat->itens[i].id, 
                cat->itens[i].nome, cat->itens[i].unidade);
    }
    fclose(f);
}
    \end{lstlisting}
    \small \textbf{Parsing:} No carregamento, usamos \texttt{strtok} para decompor as linhas e reconstruir as estruturas em memória.
\end{frame}

% --- SLIDE 6: API em C (Comunicação) ---
\begin{frame}[fragile]{Comunicação via JSON e Stdin/Stdout}
    O motor em C atua como um serviço, comunicando-se com o Node.js via fluxo de texto padrão.
    \begin{lstlisting}[language=C, title=Resposta da API: api.c]
void respond_fail(const char *msg) {
    printf("{\"ok\":false,\"error\":\"%s\"}\n", msg);
    fflush(stdout); // Crucial para o Node.js ler em tempo real
}
    \end{lstlisting}
    \begin{itemize}
        \item \textbf{Interoperabilidade:} O motor processa comandos e responde JSON.
        \item \textbf{Desacoplamento:} A lógica de dados (C) é independente da interface visual.
    \end{itemize}
\end{frame}

% --- SLIDE 7: Arquitetura do Sistema ---
\begin{frame}{Visão Sistêmica: Do Clique ao Bit}
    \begin{center}
        \begin{tikzpicture}[node distance=2cm, auto, >=stealth]
            \tikzstyle{box} = [rectangle, draw, fill=maincolor!10, text centered, rounded corners, minimum width=2.5cm, minimum height=1cm]
            \node[box] (web) {Browser (JS)};
            \node[box, right of=web, xshift=1.5cm] (node) {Server (Node)};
            \node[box, right of=node, xshift=1.5cm] (c) {Engine (C)};
            \node[box, below of=c] (file) {Data (.txt)};
            
            \draw[->, thick] (web) -- node {HTTP} (node);
            \draw[->, thick, bend left] (node) -- node {Stdin} (c);
            \draw[->, thick, bend left] (c) -- node {Stdout} (node);
            \draw[->, thick] (c) -- node {I/O} (file);
        \end{tikzpicture}
    \end{center}
    \small \textbf{Integração:} As Estruturas Lineares em C garantem a segurança que o frontend moderno apenas exibe.
\end{frame}

% --- SLIDE FINAL ---
\begin{frame}[c]
    \centering
    \Huge \textbf{Demonstração Prática}
    
    \vspace{0.5cm}
    \Large Visualizando a Fila e o Mecanismo de Rollback
    
    \vspace{2cm}
    \small \textsl{Obrigado pela atenção! Perguntas?}
\end{frame}

\end{document}
