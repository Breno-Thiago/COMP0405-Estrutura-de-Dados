\documentclass[aspectratio=169]{beamer}

\usepackage[utf8]{inputenc}
\usepackage[T1]{fontenc}
\usepackage[brazil]{babel}
\usepackage{amsfonts,amsmath,oldgerm}
\usepackage{listings}
\usepackage{xcolor}
\usepackage{tikz}

% --- THEME AND COLORS ---
\usetheme{_statale}
\usefonttheme[onlymath]{serif}

% Cores do VS Code
\definecolor{codegreen}{rgb}{0,0.6,0}
\definecolor{codegray}{rgb}{0.5,0.5,0.5}
\definecolor{codepurple}{rgb}{0.58,0,0.82}
\definecolor{backcolour}{rgb}{0.95,0.95,0.92}

\lstset{
    language=C,
    backgroundcolor=\color{backcolour},   
    commentstyle=\color{codegreen},
    keywordstyle=\color{magenta},
    numberstyle=\tiny\color{codegray},
    stringstyle=\color{codepurple},
    basicstyle=\ttfamily\scriptsize,
    breakatwhitespace=false,         
    breaklines=true,                 
    captionpos=b,                    
    keepspaces=true,                 
    numbers=none,                    
    showspaces=false,                
    showstringspaces=false,
    showtabs=false,                  
    tabsize=2,
    frame=single
}

% --- METADATA ---
\title{Sistema de Gerenciamento de Cozinha Transacional}
\subtitle{Estrutura de Dados - UFS}
\course{Ciência da Computação}
% Autor curto para o rodapé não estourar
\author[Grupo 6: Helen, João, Breno, Gabriel, Caio, Lucas]{Helen Bispo, João Simões, Breno Santos, Gabriel Bernardo, Caio Leite, Lucas Cavalcante}
\IDnumber{2026}

% --- LOGOS ---
% Redefine logo do topo esquerdo (DCOMP)
\pgfdeclareimage[height=0.9cm]{logo}{assets/dcomp.png}
\renewcommand{\@logo}{logo}

% Adiciona logo da UFS no rodapé direito
\addtobeamertemplate{footline}{}{%
    \begin{tikzpicture}[remember picture,overlay]
        \node[anchor=south east, xshift=-0.5cm, yshift=0.2cm] at (current page.south east) {
            \includegraphics[height=0.8cm]{assets/ufs.png}
        };
    \end{tikzpicture}
}

% Fundo do slide
\titlebackground*{assets/background}

\begin{document}

\maketitle

\footlinecolor{maincolor}

% --- SLIDE 1: O Que Faz? ---
\begin{frame}{Definição do Sistema}
    \begin{columns}
        \begin{column}{0.6\textwidth}
            \begin{itemize}
                \item \textbf{Objetivo:} Gerenciar o fluxo completo de uma cozinha industrial.
                \item \textbf{Fluxo:} Cadastro de Insumos $\rightarrow$ Montagem de Receitas $\rightarrow$ Controle de Estoque $\rightarrow$ Fila de Pedidos.
                \item \textbf{Diferencial:} Motor transacional em C.
                \item \textbf{Garantia:} Integridade total dos dados (Commit/Rollback).
            \end{itemize}
        \end{column}
        \begin{column}{0.4\textwidth}
            \begin{alertblock}{Problema Real}
                Se acabar a farinha no meio de um bolo, o que acontece com os ovos já quebrados?
                \textbf{Rollback!}
            \end{alertblock}
        \end{column}
    \end{columns}
\end{frame}

% --- SLIDE 2: Mapeamento Técnico (Visual Melhorado) ---
\begin{frame}{Mapeamento de Estruturas de Dados}
    \begin{columns}[t]
        \begin{column}{0.32\textwidth}
            \begin{block}{Entidades (Structs)}
                Modelagem de \textbf{Ingredientes}, \textbf{Receitas} e \textbf{Pedidos} com \textit{typedef struct}.
            \end{block}
            \begin{block}{Memória (Ponteiros)}
                Uso de \texttt{malloc} e \texttt{free} para gestão eficiente.
            \end{block}
        \end{column}
        
        \begin{column}{0.32\textwidth}
            \begin{block}{Armazenamento (Array)}
                \textbf{Catálogo} e \textbf{Estoque} usam vetores dinâmicos para acesso rápido ($O(1)$).
            \end{block}
            \begin{block}{Composição (Lista)}
                \textbf{Receitas} são Listas Encadeadas de ingredientes (tamanho variável).
            \end{block}
        \end{column}
        
        \begin{column}{0.32\textwidth}
            \begin{block}{Fluxo (Fila FIFO)}
                \textbf{Pedidos} são processados na ordem de chegada.
            \end{block}
            \begin{alertblock}{Segurança (Pilha LIFO)}
                \textbf{Rollback} usa pilha para desfazer operações falhas.
            \end{alertblock}
        \end{column}
    \end{columns}
\end{frame}

% --- SLIDE 3: Structs e Ponteiros ---
\begin{frame}[fragile]{Structs e Alocação Dinâmica}
    A base de tudo. Usamos structs para agrupar dados e ponteiros para conectar nós.
    \begin{lstlisting}[language=C, title=Exemplo: Nó de Ingrediente (ingredientes.h)]
typedef struct NoIngrediente {
    int id_ingrediente;
    float quantidade;           // Quantidade necessaria
    struct NoIngrediente* prox; // Ponteiro para o proximo
} NoIngrediente;
    \end{lstlisting}
    \vspace{0.2cm}
    \textbf{Chave da Implementação:} O uso de \texttt{prox} permite que uma receita tenha infinitos ingredientes sem realocar memória contígua.
\end{frame}

% --- SLIDE 4: Arrays Dinâmicos ---
\begin{frame}[fragile]{Arrays Dinâmicos (Estoque)}
    Para o catálogo e estoque, onde a busca por ID precisa ser rápida.
    \begin{lstlisting}[language=C, title=Estrutura de Controle (catalogo.h)]
typedef struct {
    IngredienteBase *itens; // Array alocado via malloc
    size_t qtd_atual;
    size_t capacidade;      // Capacidade total antes do realloc
} CatalogoIngredientes;
    \end{lstlisting}
    \small
    \textbf{Por que Array?} Acesso direto \texttt{itens[i]} é muito mais rápido que percorrer lista para consultas frequentes.
\end{frame}

% --- SLIDE 5: Lista Encadeada ---
\begin{frame}[fragile]{Listas Encadeadas (Receitas)}
    \begin{columns}
        \begin{column}{0.5\textwidth}
            \textbf{Cenário:}
            \begin{itemize}
                \item Bolo: 5 ingredientes.
                \item Café: 2 ingredientes.
                \item Feijoada: 15 ingredientes.
            \end{itemize}
            \vspace{0.5cm}
            \textbf{Solução:} Lista Encadeada permite inserção dinâmica sem desperdício de memória.
        \end{column}
        \begin{column}{0.5\textwidth}
            \begin{lstlisting}[language=C, title=Função de Busca na Lista]
NoIngrediente* ing_buscar(NoIngrediente* cabeca, int id) {
    NoIngrediente* atual = cabeca;
    while (atual != NULL) {
        if (atual->id_ingrediente == id)
            return atual;
        atual = atual->prox;
    }
    return NULL;
}
            \end{lstlisting}
        \end{column}
    \end{columns}
\end{frame}

% --- SLIDE 6: Fila FIFO ---
\begin{frame}[fragile]{Fila de Pedidos (FIFO)}
    Gerencia a ordem de produção. O primeiro cliente a pedir é o primeiro a ser servido.
    \begin{lstlisting}[language=C, title=Estrutura da Fila (pedidos.h)]
typedef struct {
    NoPedido* inicio; // Onde sai (dequeue)
    NoPedido* fim;    // Onde entra (enqueue)
} FilaPedidos;
    \end{lstlisting}
    \vspace{0.2cm}
    \textbf{Lógica:} Novos pedidos entram no `fim`. O processador pega sempre do `inicio`.
\end{frame}

% --- SLIDE 7: Pilha e Rollback (O Diferencial) ---
\begin{frame}[fragile]{Pilha e Rollback (LIFO)}
    A \textbf{Pilha} salva o estado para desfazer operações em caso de erro (falta de estoque).
    \begin{columns}
        \begin{column}{0.5\textwidth}
            \begin{block}{Algoritmo de Rollback}
                1. Tenta retirar ingrediente.
                2. Sucesso? \textbf{PUSH} na pilha.
                3. Falha? \textbf{POP} tudo e devolve.
            \end{block}
        \end{column}
        \begin{column}{0.5\textwidth}
            \begin{lstlisting}[language=C, title=Função de Desfazer]
void rb_desfazer(PilhaRollback* pilha, Estoque* est) {
    int id; float qtd;
    while (rb_pop(pilha, &id, &qtd)) {
        // Devolve ao estoque
        est_adicionar(est, id, qtd); 
    }
}
            \end{lstlisting}
        \end{column}
    \end{columns}
\end{frame}

% --- SLIDE 8: Arquitetura e Interface ---
\begin{frame}{Arquitetura Híbrida}
    \begin{columns}
        \begin{column}{0.5\textwidth}
            \begin{itemize}
                \item \textbf{C (Core):} Regras de negócio, estruturas e persistência.
                \item \textbf{Node.js:} Bridge via \textit{stdin/stdout}.
                \item \textbf{Web:} Interface visual para ver as estruturas "vivas".
            \end{itemize}
        \end{column}
        \begin{column}{0.5\textwidth}
            \begin{block}{Fluxo de Dados}
                Clique no Site $\rightarrow$ JSON $\rightarrow$ Node.js $\rightarrow$ \texttt{./cozinha\_api} $\rightarrow$ Structs/Ponteiros
            \end{block}
        \end{column}
    \end{columns}
\end{frame}

% --- SLIDE FINAL ---
\begin{frame}[c]
    \centering
    \Huge \textbf{Demonstração Prática}
    
    \vspace{0.5cm}
    \Large Vamos ver o \textit{Rollback} visual e o motor em ação!
    
    \vspace{2cm}
    \small \textsl{Obrigado pela atenção! Perguntas?}
\end{frame}

\end{document}
